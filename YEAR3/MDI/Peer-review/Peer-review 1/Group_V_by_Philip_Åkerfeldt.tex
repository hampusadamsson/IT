\documentclass[11pt,a4paper]{article}
\usepackage[utf8]{inputenc}
\usepackage{fullpage}
\usepackage{amsmath}
\usepackage{amsfonts}
\usepackage{amssymb}
\usepackage[hidelinks]{hyperref}


\title{\textbf{HUMAN COMPUTER INTERACTION \\(course 1MD016) \\
    Uppsala University -- Spring 2015 \\
    Report for Peer Review $1$ \\
    Review of Group V  % replace n by 1, 2, or 3,
    }}              % replace t by your team number

\author{
Philip Åkerfeldt\\
\textup{Dept. of Information Technology}\\
\textup{Uppsala University}\\
\textup{Uppsala Sweden}\\
\textup{Philip.Akerfeldt.4987@student.uu.se}
}

%\date{Month Day, Year}
\date{\today}

\begin{document}
\maketitle
\newpage
\tableofcontents
\pagebreak


\section{Structure}

%\begin {enumerate}
%\item[•] Is the information divided into sections appropriately?
%\end {enumerate}

After some analysing I find the structure to be fairly easy to follow. This is mainly because it follows the template given for this report and there are no derivations from the template for all I can see. 
The language that was used in the report was sufficient to give the reader a good understanding of the problems with the analysed system. There were however some grammatical mistakes in the text and this lead to the text being somewhat confusing at times. The mistakes mentioned were not many and most of them are not worth mentioning at all but the ones that did lead to some confusion will be mentioned below. \\

\textbf{Abstract}
\begin{quote}
"…. equipment, but none of these is studied …" 
\end{quote}
 Should be \underline{are} due to the fact that there are multiple things spoken of in the text.\\

\textbf{Introduction}
\begin{quote}
"...but this project focuses mainly on the system used by students …"
\end{quote}
Should in my opinion be \underline{only} because you don’t focus on the other universities. \underline{Mainly} should have been used if there would have been a primary focus on students from Uppsala University but you still mentioned students from other universities.\\

\textbf{User Groups}
\begin{quote}
"The system is only available to the users that are students at Uppsala university."
\end{quote}
Earlier in the document you mention that students from other universities can use the service and this sentence contradicts the statement from the previous section in my opinion. I do understand that you mean that you only choose Uppsala University but the sentence doesn't really say that. \\

\textbf{Use Context}
\begin{quote}
"A personal meeting room is, in these situations, a extremely beneficial resource ..."
\end{quote}
Should in my opinion be \underline{an} instead of \textit{a}.

%\begin{enumerate}
%\item[•] Does it provide you with a good understanding of the system that has been evaluated? Can the evaluation be repeated by using the method-section as a guide? 
%\end{enumerate}

The text describes the problems that were found in the system in a good way. But some visual representation of the issues discussed would have been a nice feature. By visual representation I mean pictures of the system that was analysed. This would have made the report more intuitive and less cryptic at some parts. The lack of visual representation does not, however, make the report less understandable but it would have contributed a better connection to the problems that were discussed.\\
The method-section describes the process really well and I find that the section could be used by other analysts to perform a similar analysis.\\ 

%\begin{enumerate}
%\item[•] How can the structure be improved? 
%\end{enumerate}
The structure of the report won't need much improvement since it follows the given template. Therefore I see no immediate way that the report could be better structured. 

\section{Cohesiveness}
%\begin{enumerate}
%\item[•] Is there a logical flow between the sections? Point out any gaps in the flow or parts that are unclear or perhaps missing.
%\end{enumerate}
The report has a logical flow in my opinion and there are no major gaps that needs mentioning.


\section{Content}
%\begin{enumerate}
%\item[•] Is the evaluation conducted in a satisfactory manner? 
%\end{enumerate}
The evaluation done by the analysers were conducted really well. I base this on the method that was chosen by the analysing group which was a really good one. The combination of easy ratings to personal opinions make a good base for understanding the problems. If the group would have chosen only a survey instead of observing and interviewing the participants they would not have gotten the sufficient data to analyse the system and therefore not been able to do a good a analysis. The observation of participants using the system made the group able to draw their own conclusions and make their own evaluations of the problems. This combined with the opinions of the participants completed a picture of the problems with the system. \\
\newline
Since the questions that were asked to the participants were good and thorough the amount of data should be conclusive. The group conducted interviews over a broad spectrum of people and this lead to a sufficient amount of data being collected. One downside of having a lot of questions concerning personal opinions is that they can be difficult to combine into a clear picture of the problems with the system. But since the system clearly had some major drawbacks that most users found quite fast this would not have been a huge problem for the analysers.\\
	An slight improvement of the evaluation could have been done using more rating-based questions in order to get a simpler picture of the problems. This was not direly needed since there already were quite a few rating-based questions. 




\end{document}