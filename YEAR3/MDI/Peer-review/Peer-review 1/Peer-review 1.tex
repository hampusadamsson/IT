\documentclass[11pt,a4paper]{article}
\usepackage[utf8]{inputenc}
\usepackage{fullpage}
\usepackage{amsmath}
\usepackage{amsfonts}
\usepackage{amssymb}

\title{\textbf{Database Design (course 1MD016) \\
    Uppsala University -- Spring 2015 \\
    Report for Peer Review $1$  % replace n by 1, 2, or 3,
    }}              % replace t by your team number

\author{Philip Åkerfeldt}  % replace by your name(s)

%\date{Month Day, Year}
\date{\today}

\begin{document}
\maketitle
\newpage
\tableofcontents
\pagebreak


\section{Structure}

\begin {enumerate}
\item[•] Is the information divided into sections appropriately?
\end {enumerate}

I find the structure to be easy to follow. This is mainly because it follows the template given for this report and there are no derivations from this template.
Is the language acceptable? 
I find that the languge that was used in the report was sufficient to give the reader a good understanding of the problems with the analyzed system. There were however some gramatical mistakes in the text and this lead to the text being somewhat confusing at times. The mistakes mentioned were not many and most of them are not worth mentioning but the ones  that lead to some confusion will be mentioned below. \\

\textbf{Abstract}
\begin{quote}
"…. equipment, but none of these is studied …" 
\end{quote}
 Should be \underline{are} due to the fact that there are multiple things spoken of (they are).\\


\textbf{Introduction}
\begin{quote}
"...but this project focuses mainly on the system used by students …"
\end{quote}
Should in my opinion be \underline{only} because you dont focus on the other universities. \underline{Mainly} should have been used if there would have been a primary focus on Uppsala University but you still mentioned some other universities.\\

\textbf{User Groups}
\begin{quote}
"The system is only available to the users that are students at Uppsala university."
\end{quote}
Earlier in the document you mention that everyone on the university can use the service and the sentence contradicts the previous statement in my opinion. \\

\textbf{Use Context}
\begin{quote}
"A personal meeting room is, in these situations, a extremely beneficial resource ..."
\end{quote}
Should in my opinion be \underline{an} instead of \textit{a}.

\begin{enumerate}
\item[•] Does it provide you with a good understanding of the system that has been evaluated? Can the evaluation be repeated by using the method-section as a guide? 
\end{enumerate}

The text describes the problems that were found in a good way. But som visual representation of the issues discussed would have been a nice feature. By visual representation i mean pictures of the system that was analyzed. This would have made the report more intuitiv and less cryptic at some parts. The lack of visual representation does not, however, make the report less understandable but it would have contributed a better connection to the problems that were discussed. 

\begin{enumerate}
\item[•] How can the structure be improved? 
\end{enumerate}

Since the report follows the template that was given by the project handler I see no immidiate way that the report could be better structured. 

\section{Cohesiveness}
\begin{enumerate}
\item[•] Is there a logical flow between the sections? Point out any gaps in the flow or parts that are unclear or perhaps missing.
\end{enumerate}



\section{Content}
\begin{enumerate}
\item[•] Is the evaluation conducted in a satisfactory manner? 
\end{enumerate}

\begin{enumerate}
\item[•]Is the chosen method appropriate for studying users and usability problems of System X? 
\end{enumerate}

The method that was chosen by the analyzing group was in my own opionion a very good one.The combination of esy ratings to personal opinions make a good base for understanding the problems. If the group would have chosen a survey instead of observing and inteviewing the participants they would not have gotten the sufficient data to analyze the system and therefore not been able to do a good a analysis. The observation of participants using the system made the group able to make their own assumptions and evaluations of the problems. This combined with the opionions of the participants completed a picture of the problems with the system. 

\begin{enumerate}
\item[•]What are the strengths of the evaluation and what can be improved? 
\end{enumerate}

Since the questions that were asked to the participants they amount of data should have been conclusive. They conducted inteviews over a broad spectrum of people and this lead to a sufficent amount of data. One downside of having a lot of questions concering personal opionins is that they can be difficult to combine into a clear picture of the problems with the system. But since the system clearly had some major drawbacks that most users found quite fast this would not have been a huge problem for the analyzers.
	An slight improvement of the evaluation could have been done using more rating-based questions in order to get a simpler picture of the problems. This was direly needed since there already were quite a few rating-based questions. 




\end{document}