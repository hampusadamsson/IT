\documentclass[a4paper,11pt]{article}
\usepackage{fullpage}

\usepackage[utf8]{inputenc}
\usepackage[british]{babel}

\usepackage{amsmath}
\usepackage{amssymb}
\usepackage{amsthm}
\usepackage{color}
\usepackage{float}
\usepackage{listings}
\usepackage{fontenc}
\usepackage[hidelinks]{hyperref}

\usepackage{multirow}


\title{\textbf{Database Design, 5 C (Course 1DL300) \\
    Uppsala University -- Spring 2015 \\
    Report for Peer Review}}

\author{
Oliver Campeau\\
\textup{Dept. of Information Technology}\\
\textup{Uppsala University}\\
\textup{Uppsala Sweden}\\
\textup{Oliver.Campeau.2983@student.uu.se}
\and
Mikael Sernheim\\
\textup{Dept. of Information Technology}\\
\textup{Uppsala University}\\
\textup{Uppsala Sweden}\\
\textup{Mikael.Sernheim.2899@student.uu.se}
\and
Philip \AA kerfeldt\\
\textup{Dept. of Information Technology}\\
\textup{Uppsala University}\\
\textup{Uppsala Sweden}\\
\textup{Philip.Akerfeldt.4987@student.uu.se}
}


\date{\today}

\begin{document}
\maketitle
\newpage
\tableofcontents
\pagebreak


\section{Overall impression, context $\&$ content}
The overall impression of the report is very good. The report grants the reader a comforting flow which makes the report easier to handle and makes the content of the report easier to digest. The re-designers have pointed out the weaknesses of the previous database and the suggested and implemented solutions for fixing, alt. improving, these weaknesses are structured well. 

The content of the report is a thoroughly display on what actions that have been made on the old database and why. The report makes sure that the reader can't misunderstand how the problems were solved. 

\section{Structure and disposition}
The structure of the report is easy to follow and there are no "gaps" in the report that needs mentioning. Some minor problems could however been easily fixed and should therefore not be ignored. One of the things that were noticed and that ends up under the previously mentioned category is the \textit{Table 1a} where the first cell breaks the text since the cell is not wide enough. This could have easily been fixed and would not have sent out a negligent impression. 

Another thing that was found was in the EER-diagram of section 3. The are multiple attributes with names that implies that it is multiple valued. An example of this is the attribute "Multiple Scores", where the second part of the name (..Scores") is plural and therefore suggests it is/can be multi-valued.


\section{The language}
The language used in the report is both structured and it follows an academic template which is undoubtedly a comforting feature since it provides the reader with a sense of security and respect for the designers.
The level of the language used is for the most part not hard at all but this attribute is not to be confused with a trite language. There were however at least two times when a word actually had to be looked up in a dictionary. This is something that should not have to be done since the report aims to "sell" a redesign of a current system and not to confuse the reader who may or may not be aware of the meaning of the word.

A nice feature added by the group is the explanation of the scientific terminology used is the report. This gives the reader, who is with great certainty not the previous developer of the database, a high-level understanding of what is to be done with the system in order to improve it. This is not something that is done for every part or every sentence which should be done considering that the receiver is not a programmer at all. 




\end{document}