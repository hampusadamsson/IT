%%
%%  Template for documentation in LaTeX
%%  Tobias Wrigstad, tobias@dsv.su.se, 2002-11-19
%%
%%  Most commands used this sample document should be 
%%  understandable from how they are used here. They are
%%  further explained in the document `The not so short
%%  guide to LaTeX' by Tobias Oetiker.For the benefit 
%%  of foreign students, the comments in this file will
%%  be written in English. 
%%
\documentclass[a4paper,11pt,twoside,titlepage,openany]{book}
%%  Declares that the current document shall use the book-template
%%  which meant that it will have headers, use chapters for partitioning
%%  the text etc. 
%%  Target paper size is A4 paper, text is 11 pt if not explicitly 
%%  altered, it will be printed on two-sided paper and it will use
%%  a special title page for printing title, author and date. 
%%  `Openany' specifies that a chapter can begin on both right and left
%%  pages. 

\usepackage[latin1]{inputenc}
%%  The use of packages in LaTeX is quite analogous with packages in 
%%  Java. The input encoding packages enables LaTeX to use non-English
%%  code pages (teckentabell). In this case latin1, to enable the use
%%  of Swedish characters such as �, �, � and accented characters such
%%  as � without special treatment. 

\usepackage[swedish]{babel}   %%  Remove when writing in english!
%%  Load Swedish hyphenation and tell LaTeX that we are writing in 
%%  Swedish. This makes LaTeX print 'kapitel' instead of 'chapter' etc.

\usepackage[dvips]{graphicx}
%%  Load support for special graphics, such as UML diagrammes etc. 
%%  Importing graphics in LaTeX is a little tricky so please assign 
%%  someone to learn to master it in the beginning of your project. 
%%  Note that only .EPS-files (encapsulated post script) may be 
%%  used! 

\usepackage{url}
%%  Defines the command \url{} that can be used to typeset url:s
%%  in text

%%  Now we will declare the title and author of this document. 

\title{\LaTeX{} at DSV2:PVT} % set the document's title

\author{Tobias Wrigstad \\ % the \\ command is a line break
  \texttt{tobias@dsv.su.se}} % \texttt{...} gives typewriter text

\date{\today} % set the document's date to today's date

%%  Here ends the declaration section, the so-called 'preamble'. Now, 
%%  the document text begins. 


\begin{document}
%%  The beginning of the document 
%%
\maketitle
%%  Print the title, the author and the date
%%  Since we specified titlepage above, this will be printed on 
%%  a single page with no other text in the beginning of the document

\chapter{Introduction}
%%  Declares the beginning of a new chapter. The book-template REQUIRES 
%%  the use of chapters, otherwise section numbering will be erroneous. 
%%  Since we specified that the document should be written in swedish, 
%%  'Kapitel 1' will be printed at the top of the page. 

\section{What is this LaTeX, anyway?}
%%  No text should be placed outside a section. This generates a section
%%  head in bold face with section number 1.1 (chapter 1, section 1). 

Now, I can begin to write some text. Note that the text automatically 
becomes size 11 pt because of the declaration in the preamble. Text is 
automatically justified (raka h�ger- och v�nstermarginaler) and new
paragraphs are automatically indented. Note (in the code) how linebreaks 
and double                               spaces are ignored by \LaTeX{}. 

A new paragraph is started by an empty line in the text. Note the 
indentation. This is clearly visible from the source code. 
\emph{N.B. You should read this document in parallel with the source code!}

\subsection{Text formatting}
%%  Generates a subsection with number 1.1.1
%%
Text can be \emph{emphasized} (which more or less means \textit{italicized} 
in the book template). Text can also be \textbf{bold faced} or in 
\textsc{Script Capitals} or without serifs i.e. \textsf{sans serif}. 

An easy way to show code snippets is to use the \texttt{verbatim} (note how
I switched to typewriter text just for one word or phrase) environment: 

%% Begins preformatted text
\begin{verbatim}
This text is preformatted, meaning
that linebreaks
are 
    not 
        ignored as well as 
double           spaces (%% comments are not ignored)
\end{verbatim}
%% Ends preformatted text

In the preformatted text section, text is written exactly as it is typed
in the source code. Line breaks are \emph{not} automatically inserted when 
the end of the line is reached. Instead text can flow into the right margin
or even out of the page! Thus, lines must be broken by hand in the 
\texttt{verbatim} environment. 

\subsubsection{The next level of sectioning lacks a number} 
Thus, it will not appear in the table of contents. 

\section{The use of tables}
First: note how this section's name is showed in the header. All
pages (except the first page of every chapter and the title page)
automatically becomes a header with a section name and number.

You may wish to use tables in your project documentations. Tables are
quite easily created. Below, a simple sample table is shown. For a more 
comprehensive documentation of how to create tables, see `The not so short 
guide to \LaTeX'. %% A good exercise: add a reference to that doc here!

The table below is generated with this code:
\begin{verbatim}
\begin{tabular}{|l|r|c||p{1.5cm}|}
\hline
left justified & right justified & 
centered & width specified \\
\hline
\hline
\end{tabular}
\end{verbatim}

\begin{tabular}{|l|r|c||p{1.5cm}|} 
%%  Note that a table is called a tabular! 
%%  The character | declares that a vertical line should separate 
%%  two columns. || means, two vertical lines etc. 
%%  The character l creates a left justified column with a width just
%%  enough to contain the longest line of text in the column. 
%%  The character r is analog with l but text is right justified. 
%%  The character c is analog with l but text is centered. 
%%  p{1.5cm} generates a column with width 1.5cm with left justified text.
\hline
%%  Generated a horizontal line
left justified & right justified & centered & width specified \\
%%  The character & skips to next column
%%  \\ ends the current row, all rows must be ended with \\
\hline
\hline
\end{tabular}


Text may be formatted with italics, bold face etc. inside a table. 
However, as we can clearly see from the appearance of this page, the
placing of the table is not correct. It almost overlaps with the text
of this paragraph. Tables should always be placed in a special environment
called a \emph{float}. A float contains a table or a figure and ensures
that it is placed correctly in the document. If there is not enough space
on a page to place a float, the float \emph{floats} over to the next 
page, hence the name. Now, we place the table inside a float:

\begin{table}[h] 
%%  Creates a float that contains a table
%%  The character h read `here' and specifies that the float should appear
%%  at the same place in the text as where it was typed. 
%%  Other options are 't' for top of page, 'b' for bottom of page and 'p'
%%  for a special floats page, i.e. an empty page with only the table on
%%  (and other floats that might fit onto the same page)
%%
\begin{center} 
%%   Centers the table in the float
%%
\begin{tabular}{|l|r|c||p{1.5cm}|}
\hline
left justified & right justified & centered &
width\newline specified \\
\hline
\hline
\end{tabular}
\end{center}
%%  End the centering
%%
\caption{My table}
%%  Gives a explanatory text/a name to the table in the float
\label{tab:my-table}
%%  See the cross referencing section for above command
\end{table}
%%  End the table

Much better, don't you think? Note that we have also given the
float a caption. Since the float was a table float (examine the code), 
the caption automatically becomes the text `Tabell'. This would not
be possible if we did not place the table in a float since then, 
\LaTeX{} cannot determine where the table and the explanatory text
begins or ends. 

\begin{table}[t] 
\begin{center} 
\begin{tabular}{c|p{10cm}}
\hline
\textsf{c} & \textsf{p} \\
\hline
1      &  lines inside a column with a specified width will be broken
        automatically if their length exceeds the width of the 
        column. This is not the case for centered, left justified
        or right justified columns where lines must be broken by
        hand \\
10    & Ten \\
1000  & Thousand \\
\hline
\end{tabular}
\end{center}
\caption{This table is autmatically placed at the top of the page 
where it appears in the code} 
\end{table}

\section{Figures}
\label{includegraphics}
%%  Labels are used to create references in a document, labels will
%%  be explained later
%%
Figures should also be placed in floats and be given captions. Figures
should contain diagrammes etc. Figures are included with the use of 
this command: 

\begin{verbatim}
\includegraphics[options]{name of .EPS file}
\end{verbatim}

Where the options are a list of keys with values separated with commas:

\begin{description}
%%  This is a list, lists will be explained in a little while
%%
\item[scale=] 
A scale factor from in per cent.

\item[angle=] 
An angle

\item[bb=] 
Creates a \emph{bounding box}, i.e. a rectangle within the included 
image. Everything outside the bounding box is omitted. A bounding box
is specified by four values, first the x and y values of the lower left
corner and then the x and y values of the upper right corner. 

\item[clip=] 
True or false. Specifies if the the contents outside of the bounding box
should be omitted or not. 

\end{description}

The figures on the next page were imported with the following commands: 

\begin{verbatim}
\includegraphics[scale=.25]{sample.eps}
\includegraphics[scale=.50]{sample.eps}
\includegraphics{sample.eps}
\includegraphics[angle=45]{sample.eps}
\includegraphics[bb=4mm 0mm 17mm 17mm, clip=true]{sample.eps}
\end{verbatim}

They all reside in the same float. The last image is clipped by the use of 
a bounding box. Normally, using a bounding box is not necessary. 

\begin{figure}[!t]
%%   This is a figure float, it works exactly as a table float but writes
%%  'Figur' instead of 'Tabell' in the caption. 
\begin{center}
\includegraphics[scale=.25]{sample.eps}
\includegraphics[scale=.50]{sample.eps}
\includegraphics{sample.eps}
\includegraphics[angle=45]{sample.eps}
\includegraphics[bb=4mm 0mm 17mm 17mm, clip=true]{sample.eps}
\end{center}
\caption{The same figure included five times}
\end{figure}

\section{Lists}
\LaTeX{} supports the generation of lists in many ways; there are bullet 
lists, enumerated lists, descriptive lists etc. Below, the most common lists
are shown. Look at the source code to see how they are generated. 

\subsubsection{Bullet list}
\begin{itemize}
\item 
In a bullet list, all list items are preceeded by a bullet, $\bullet$

\begin{itemize}
\item 
Lists can be nested, i.e. a list item can contain a new list (even 
of a different kind). Note how the appearence of the nested list changes. 

\end{itemize}

\item 
New items are automagically indented 
\end{itemize}

\subsubsection{Enumerated list}
\begin{enumerate}
\item 
In a enumerated list all list items are preceeded with a number. 

\begin{enumerate}
\item 
In nested enumerations, different enumeration styles are used. 

\begin{enumerate}
\item 
A maximum of four levels of nesting is supported. 

\end{enumerate}

\end{enumerate}

\item 
New items are automagically indented 
\end{enumerate}

\subsubsection{Descriptive list}
\begin{description}

\item[Descriptive list] A descriptive list is a list where each
item has a name written in bold face.  Note how a descriptive list
was used to show the options of includegraphics in Section
\ref{includegraphics}, page \pageref{includegraphics}.
%%  \ref and \pageref are used to refer to different places inside a document
%%  their usage will be explained in a little while
\end{description}

\section{Floats}
Just a short note on floats. Always place tables and figures in
floats.  Use table floats for tables and figure floats for
figures. Look at the source code on lines 162 and 265. Read the
comments and experiment!

\section{Structuring your documents}
This purpose of this document is to be a guide to formatting
documents in \LaTeX. To be readable codewise -- it is very badly
structured (even technically). When you write your own documents
in \LaTeX{} you must not be as sloppy as this.

\subsection{File inclusion}
A document can be divided into several different files to
facilitate different people working at the same document at the
same time. Files are included with the command:
%
\begin{verbatim}
\input{file name} 
\end{verbatim}

File name must be without spaces and file extension (extension
\emph{must} be \textsc{tex}). If a file is placed in another
directory, the entire path from the main file to the included file
must be specified, \emph{e.g.}\ ``\texttt{../../../myfile}''.

\subsection{Cross referencing}
In large documents (such as those that you are about to write) 
it is often necessary to refer to a different place in the
text. This is easy in \LaTeX{} with the commands:
%
\begin{verbatim}
\label{name} 
\ref{name}
\pageref{name}
\end{verbatim}

The \verb.\label. command associates a name with a place in the
document text.  This can be issued almost everywhere. By using
\verb.\ref., the number of the section (or float or list item
number etc.) where the label appeared is printed;
\verb.\pageref. prints the number of page where the label appeared
is printed.  Depending on where the \verb.\label. was issued,
different numbers appear.  Experiment! Look at lines 220 and 329
in the source code for an example.

\subsection{Citation with the bibliography}
Hopefully, you will divide your project documentation into several
different documents. Needless to say, you will sometimes (or quite
often) need to refer between these documents, e.g. to say that X
is explained in Y etc.

To refer to external documents, you must first create a
bibliography (litteraturf�rteckning) of documents that you
wish to refer. A bibliography is included at the end of this
document (look at the source code to see how it is constructed).

To refer to an external document, the command \verb.\cite. is
used. It works analogously with \verb.\ref. except that there is
no corresponding label. Instead, the names that may be cited are
specified by the bibliography at the end of the document. The
bibliography at the end of this document contains three documents
with the names `gof-patterns', `captain69' and `template'. Either
of these names can be used by cite. For example,
\verb.\cite{captain69}. produces \cite{captain69}.
%%  Here, \cite is used with a label from the bibliography at the end
%%  of the document

In text, you usually write, `\ldots for a more detailed
description of the proxy-pattern, see \cite{gof-patterns}\ldots'

When referring to your own documents, you generally need to
specify more exactly the location of the text to which you are
referring, e.g. `\ldots for a more detailed description of the
proxy-pattern, see \cite[pp. 257-263]{gof-patterns}\ldots'

This is achieved by \verb,\cite[pp. 257-263]{gof-patterns},. The
text in the brackets can be any text, not necessarily page
numbers.


\subsection{Table of contents}
In \LaTeX{}, a table of contents is easily generated by issuing
the command

\begin{verbatim}
\tableofcontents
\end{verbatim}

The table of contents in the course documentation is automatically
generated using this command. (Exercise: insert it in this
document and see the results -- note that the \LaTeX\ compiler
will tell you to rerun the compilation at least one time!)

\section{Line breaking}
In \LaTeX{}, line breaking is a real issue. This is due to that
\LaTeX{} must obey a number of typographic rules that is embedded
in the system. In some cases, e.g. \LaTeX{} don't know how to
break a very long word, it is necessary to manually hyphenate or
break the line by hand. There are a couple of different approaches
to this. The line below illustrates one problem: \\[3mm]
%%  Note that you can specify how tall a line break should be 
%%
1)~~Thislinedoesnotcontainanyspacesandwhen{\LaTeX}triestobreakititwillhaveproblems. 
%%  The character ~ is a `hard space' that will not be compacted by LaTeX
%%   always use hard spaces (also in Microsoft Word!) when writing e.g. 1~000 kr 
%%  to avoid line breaking between 1 and 000. 
%%

\begin{description}
\item[hyhenation] 
  When \LaTeX{} doesn't know how to hyphenate a word, you can do
  it manually with a special (soft) hyphen, \verb.\-.. This will
  insert a hyphen \emph{if necessary}.  If the text is altered
  later and suddenly, the long word fits into one line without
  hyphenation, the hyphenation will not be inserted. A long word
  could have many soft hyphens to enable hyphenation at several
  different places.

\item[linebreak] 
  The command \verb.\linebreak. or \verb.\linebreak[num]. breaks a
  line and justifies the margins. If you use
  \verb.\linebreak[num]., you must specify a number between 0 and
  4 that is an indication of how much you want the line broken
  (higher number indicates higher need for line breaking). This
  enables the removal of deprecated line breaks that are no longer
  necessary due to subsequent changes in the text.

\item[newline] 
  The command \verb.\newline. breaks a line. There is a shorthand
  for \verb.\newline.  in the command \verb.\\. or
  \verb.\\[length].. The latter breaks a line but also specifies
  how much whitespace should be inserted before next paragraph.

  Sometimes, \LaTeX\ is unable to justify the right margin of a
  section, simply because there is not enough words on the for
  \LaTeX\ to insert enough space between so to give the line the
  desired length. This is signalled by the \LaTeX-compiler as an
  ``underful hbox''.

  There is one such example in this document in the last column in
  the Table \pageref{tab:my-table}.  Look in the code on the first
  table on the same page to see the use of \verb.\newline. (line
  175 in the code). You may also want to look at the output of a
  compiler, fix the faulty table and recompile. Note that,
  depending on the situation, \verb.\linebreak. might actually be
  the solution (if you want the justified margins). This command
  does not, however, stretch the length of a word. 

\end{description}

\section{Running \LaTeX}
This is a brief explanation of how to transform a \textsc{tex}
source file into a viewable/printable document.  \LaTeX{} is
invoked from the command line with the main file as single
parameter:
%
\begin{verbatim}
$ernst-hugo>latex my-file
\end{verbatim}

This will cause the file to be `compiled' (in some cases it must
be compiled several times due to forward references etc., in this
case \LaTeX{} will print a message indicating that the file should
be recompiled) and produce a viewable file called a \textsc{dvi}
file (DeVice Independent). If there are any compiler errors, these
will be reported and the compilation aborted (without producing a
\textsc{dvi} file). The \textsc{dvi} file is viewed with the
command:

\begin{verbatim}
$ernst-hugo>xdvi my-file.dvi
\end{verbatim}

Some machines may have additional (and better) programs for
viewing \textsc{dvi} files, like \texttt{kdvi} (available in the k
desktop system, \url{www.kde.org}).

After a \textsc{dvi} file is produced, another program called
\texttt{dvips} is used to produce a postscript file or feed it to
the printer. To create a postscript file from a \textsc{dvi} file
(that can be emailed to your contractor):
%
\begin{verbatim}
$ernst-hugo>dvips -ta4 -Ppdf -ooutfile.ps my-file.dvi
\end{verbatim}

This will cause the contents of my-file.dvi to be printed to the
postscript file \texttt{outfile.ps}.

\textbf{Please be careful when using \texttt{dvips}! If you forget
the \texttt{-o} flag, your prints will be spooled to the printer
instead of to a PostScript file!}

To print the contents of a \textsc{dvi} file to a printer:
%
\begin{verbatim}
$ernst-hugo>dvips -ta4 my-file.dvi
\end{verbatim}

To convert a \textsc{ps} file to a \textsc{pdf} file:
%
\begin{verbatim}
$ernst-hugo>ps2pdf13 mypsfile.ps 
\end{verbatim}

This creates a new \textsc{pdf} file names \texttt{mypsfile.pdf}.

If \texttt{ps2pdf13} is not available, but \texttt{ps2pdf} is,
then the same result can be achieved using \texttt{ps2pdf
  -dCompatibility=1.3 myfile.ps}. 

\section{How do I start?}
Copy the first 61 lines of this file and add a
\verb.\end{document}.. Strip the comments if you think that you
don't need them. This should compile just fine.  Alter the title,
author and date. Now you have a one-page \LaTeX{} document with
only the titlepage. Now, start writing between
\verb.\begin{document}. and \verb.\end{document}.. The
bibliography is easily copied from this document into your new
document and the contents of the bibitems altered\footnote{Tip:
you can use the same bibliography file for all documents and just
include it with the input command. Or better still, learn
Bib\TeX, which is a very potent powerful bibliography system for
\LaTeX.}. Good luck!

\paragraph{Comment}
On the course DSV2:PVT, all documentation managers will be
supplied with a copy of `The not so short guide to \LaTeX' and
also be given an introduction on how to run \LaTeX{} and write a
small \textsc{tex} file. It might be a good idea to assign a
project member to work out the details of \LaTeX{} that you think
you need. That way, you will produce a local expert that can be
called in to solve \LaTeX{} problems as they arise.

\begin{thebibliography}{99}
%%  Start of bibliography. It works almost exactly like a list, but with 
%%  \bibitem instead of \item. 
%%
\bibitem{gof-patterns} 
    \textit{Design Patterns, Elements of Reusable
    Object-Oriented Software}.
    Erich Gamma, Richard Helm, Ralph Johnson, John Vlissides. 
    Addison-Wesley 1995.

\bibitem{captain69} 
    \textit{Safe as Milk}. 
    Captain Beefheart, Zoot Horn Rollo, Winged Eel Fingerling, 
    Alex Snouffler, John French. 
    Budda Records 1969.

\bibitem{template} 
%%  Gives a name for citations to the following document, in this case `template'
%%  
    \textit{Title of document}. 
    List of authors in the order they appear in the document 
    Name of publisher, or other remark and year of publication. 
    Document number or version number is good to include. 
\end{thebibliography}
\end{document}

\end{document}
%%  The end of the document
%%